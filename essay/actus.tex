\documentclass[12pt]{article}

\usepackage{amsfonts}
\usepackage{amsmath}
\usepackage{graphicx}
\usepackage{lmodern}
\usepackage{hyperref}
% \usepackage[colorlinks=none,linkcolor=blue,pdfnewwindow=true]{hyperref}
\usepackage{xcolor}
\definecolor{blue}{rgb}{0.0, 0.0, 1.0}
\hypersetup{
  colorlinks   = true, %Colours links instead of ugly boxes
  urlcolor     = blue, %Colour for external hyperlinks
  linkcolor    = black, %Colour of internal links
  citecolor   = red %Colour of citations
}
\usepackage[document]{ragged2e}
\usepackage{todonotes}

\def\bi{\begin{itemize}}
\def\ei{\end{itemize}}

\setlength{\parindent}{0pt}
\usepackage[skip=10pt plus1pt, indent=40pt]{parskip}
\setuptodonotes{inline}

\title{Financial contracts on distributed ledgers}
\author{Nick Van den Broeck, Casper Association}

\begin{document}

\vspace{-1cm}
\maketitle

\tableofcontents
\newpage

\section{Introduction}

Over the last decade, the blockchain industry has been moving forward with smart
contracts, allowing DeFi to program currencies adhering to "code is law".
However, most real-world financial contracts don't fit within this framework.
Rather, they require a jurisdiction to be associated with the contracts.
Consider as an example a loan, where the loanee could decide to stop making
monthly payments. In such a case, the loaner requires tools to keep the loanee
accountable, escalating up to legal measures. In practice, our modern legal
escalation process is so effective that this escalation is an edge case. In
contrast, it is impossible to make code force the loanee to make monthly
payments or to take into account legal concepts such as bankrupcy, unless
overcollateralization is required.

As we can see from this example, in order to grow into a plethora of use cases,
the blockchain industry requires pushing beyond smart contracts into smart
financial contracts. These are on-chain records detailing financial agreements
which live within a set jurisdiction. In essence, a smart financial contract is
an agreement from which a set of future cashflows can be derived which includes
the mediation of a specific jurisdiction.

The creation of smart financial contracts rests upon two innovations: A standard
for describing financial contracts in code, and a way to extend the transaction
throughput and privacy controls on a blockchain. In this essay, we will describe
both of these innovations as well as how to combine them in order to reasonably
implement smart financial contracts.

This essay starts by explaining why one should care about smart financial
contracts in the first place. After discussing some use cases, section
\ref{actus} digs into the first innovation, the ACTUS standard on financial
contracts. Sections \ref{l2} and \label{zkp} discuss Casper's need for an L$2$
and the innovation we require for this L$2$: Zero-knowledge proofs (ZKPs). We
explain what a ZKP-based L$2$ requires and its general structure. We complete
the story in sections \ref{mvp} and \ref{vision} by exploring what such an $L2$
will look like for the Casper blockchain as an MVP and in the long-term vision.
Finally, we will present our conclusions. Technical details on the Casper
Association's implementation plans are left for a future essay.

\section{The importance of smart financial contracts}

So now to the crux of the introduction: Why should you care about integrating
ACTUS with the Casper blockchain? There are three parts to this question: Why
would the financial industry care about ACTUS, why on a blockchain, and why does
this matter for Casper?

To start things off, let's discuss why the financial industry cares about ACTUS.
The industry has significant issues. As James Grant put it, ``Progress is
cumulative in science and engineering, but cyclical in finance.'' This is a
clear sign that there is something wrong. The system seems to go through cycles
of increasing and decreasing regulation without learning the right lessons. One
of the ways to break through this issue is to increase transparency in financial
institutions and their risk both for the institutions themselves, for external
risk analysts, and for regulators. This requires two things: Standardization of
regulatory reporting for financial institutions, and for the content of the
reports to be usable as a basis for analyses. Currently, financial reports
mostly discuss the company's accounting situation, which provides meaningful
insights but prevents many forms of analysis. Rather, we should look at the
fundamentals of these institutions, i.e. their financial contracts, and use
these as a basis for reporting. Secondly, a standard is required such that
reporting of information on financial contracts is homogeneous. This will allow
anyone to run their own risk analysis on important institutions, both internally
and externally to the company, in order to keep them accountable and provide
natural feedback mechanisms against excessive risk.

Why would the financial industry want the ACTUS protocol to be integrated with a
blockchain? The answer is privacy, security and scalability.
\todo{@Mark: Could you write a paragraph here? (Auditable trust) I'm assuming
the goal here is to allow the financial institution to expose enough information
about itself so people can assure themselves they're doing well, while also
protecting the privacy of their customers and things like investment
strategies.}

Finally, why does any of this matter to the Casper community? The Casper
blockchain is very well-built, but does not have the first movers' advantage in
smart contracts-based blockchains. Therefore, Casper must do something in order
to set itself apart and provide a unique value proposition. The combination of
an exploration of zero knowledge-based L$2$ technology and attracting people
through the plethora of use cases in the financial industry, gives the Casper
community the tools to pursue significant growth while adding clear value to one
of the most important industries.

\todo{@Mark: What do you think about the last paragraph?}

\section{Use cases for ACTUS on Casper}

Risk analysis has been performed in the financial industry for a long time.
However, an important rule in software and any form of analysis, is ``garbage
in, garbage out''. Without a large amount of high-accuracy, high-relevance data
as input to the analysis, nothing meaningful can be derived. This leads us to
three problems with modern-day financial reporting.

First of all, external risk analysts and financial regulators do not have access
to raw data pertaining to financial institutions, but rather to the data
reported by these institutions. This reporting happens mostly for accounting
purposes, and its format is adapted to that goal. The data is therefore both
low-relevance from a risk analysis perspective, and much more open to being
influenced by the institution in order to manipulate the risk analysis. For
example, Silicon Valley Bank's accounting showed no issues whatsoever until very
briefly before it entirely shut down, leaving a gaping hole in the economy and
reputation of the financial industry. Therefore, we need a new way of reporting
information, not based on accounting principles but rather directly discussing
the relevant unit of information here: The financial contract.

Secondly, there is currently no homogeneity in the storing and reporting of this
high-relevance data. This makes it very difficult to run risk analyses, since
there is significantly reduced opportunity for collaboration between
institutions, as shown in figures \ref{fig:non-standard-risk-analysis} and
\ref{fig:standard-risk-analysis}. In the former figure we observe a situation
where data is stored and communicated in a heterogenous way. As you can see,
anyone who wants to run a new form of risk analysis on $N$ financial
institutions in this situation, has to write code to convert the data of $N$
financial institutions into a format he can use for his risk analysis.
Introducing the ACTUS standard would convert this unmanageable amount of work
into simply turn the ACTUS data into the input format of the new risk analysis.

\begin{figure}
  \centering
  \includegraphics[width = 0.7\textwidth]{non-standardized-risk-analysis.png}
  \caption{Non-standardized risk analysis}
  \label{fig:non-standard-risk-analysis}
\end{figure}

\begin{figure}
  \centering
  \includegraphics[width = 0.9\textwidth]{standardized-risk-analysis.png}
  \caption{Standardized risk analysis}
  \label{fig:standard-risk-analysis}
\end{figure}

\todo{@Mark Explanation of potential use cases for the system, such as for
trading and settlement in decentralized exchanges or for managing financial
contracts in traditional financial institutions}

At Casper Association, we want to provide the tooling to help financial
institutions adopt the ACTUS standand and communicate their financial
information while preserving sensible privacy options. Eventually, some
financial institutions could even decide to adopt Casper as their core software,
as a trusted, decentralized version of a mainframe.

\section{ACTUS} \label{actus}

As mentioned above, the financial industry is in need of a standard in which to
store and communicate financial information The standard must fit certain requirements:
\begin{enumerate}
  \item Express the information compactly
  \item Allow for sensible compression throughout analyses
  \item Form a solid basis for analyses by both regulators and risk analysts
  \item ideally, allow financial institutions to derive their accounting from
    the information reported in the standard
\end{enumerate}

The ACTUS standard proposes that the correct unit for such a standard is the
financial contract.
\todo{Why exactly should the financial contract be considered the atomic unit of
the global economic system?}
A financial contract is a set of terms and a set of external
dependencies\footnote{
  These are things such as stock prices.
} which lead to a set of predicted cashflows between the agreeing parties. The
ACTUS standard defines a taxonomy of financial contract, breaking down the choas
into $32$ different contract types, each with algorithmic future cashflow
generation. See \href{https://www.actusfrf.org/}{the ACTUS website} for more
information on the standard itself.

The ACTUS standard fits all of our criteria. First of all, each contract is
described as a set of terms, which are numbers, and an update function which
depends on external sources, which are numbers. Therefore, all the data we want
to store, is the contract type and a set of numbers per contract. This forms a
compact expression of the information.

Secondly, for the purpose of running a risk analysis, any contracts of the same
type and with similar terms, can be approximated by one larger contract of the
same type. This allows for significant compression in a world where millions or
even billions of contracts could exist within a single isntitution.

Thirdly, ACTUS-based reporting forms a basis for risk analysis, since all the
relevant financial information is present in order for an analyst to run a
simulation against any scenario\footnote{
  A scenario to run a risk analysis on, consists of values for all the relevant
  external sources, such as stock prices and interest rates.
}.

Finally, a firm's bookkeeping can be entirely derived from ACTUS-based
reporting, since no financial information is lost.

\section{Why we need an $L2$} \label{l2}

Before we delve into the blockchain side of this project, let us recap what we
have discussed so far. Firstly, the financial industry is in need of
transparency. Secondly, this need can be resolved by combining the ACTUS
standard on financial contracts with a purpose-built L$2$ on top of Casper. Now
it is time for the third part of the equation: Zero-knowledge proofs as a way to
build an L$2$ for casper.

Firstly one could ask why we need an L$2$ on top of Casper in order to integrate
ACTUS. The main reason for this is Casper's current throughput restriction. Any
financial institution logging its contracts on Casper would store millions of
contracts, which Casper's L$1$ simply cannot handle quickly enough. In addition,
this project requires specialized privacy controls not offered by Casper
currently.

% Why not simply integrate in L1? Because L1 is a very complex problem related
% to security and consensus. You can only optimize for so much.

ACTUS on Casper thus requires an L$2$. There are several types of L2 solutions,
including optimistic rollups, zero knowledge rollups and sidechain-based
systems. We at the Casper Association decided to focus on Zero knowledge Rollups
for a number of reasons:
\bi
  \item They seem to be the most feasible path forward.
  \item They allow more finegrained privacy control than most other options.
  \item They allow for larger scripts to be executed than with most
    alternatives, since the scripts themselves never touch the L1. Of course the
    proof size and construction time are limiting factors here.
\ei

In the next few chapters, we will create some intuition for what zero knowledge
proofs are and what constitutes a zero knowledge-based L$2$, or so-called zero
knowledge rollup.

\section{Zero knowledge proofs} \label{zkp}

Zero knowledge proofs (ZKPs) are an implementation of the idea that you can
prove you have something without revealing the thing. A common example of this
are public/private key-pairs. In this example, I can generate a key-pair and
prove to you that I have the private key without revealing it. This works as
follows. First I tell you which public key I have, which is associated with my
private key. Next, you generate some bytestring and send it to me, and I sign it
using the private key. You can then verify the signature using the public key,
of which I am claiming to have the associated private key, and thereby confirm
that I do have the private key. However, the signature itself cannot be used to
easily discover the private key itself. This is called a zero knowledge proof,
i.e. I prove to you that I have something without revealing the thing I'm
proving I have.

Zero knowledge proofs have two great properties, namely compression and privacy
control. Firstly, it is possible to generate proofs which are shorter than the
secrets they are proving\footnote{
  Note that this is provably impossible in the general and exact case. However,
  ZKPs are probabilistic, meaning that they don't bar false positives. In
  practice this is something to keep in mind when designing ZKP systems, to make
  sure this doesn't become an issue in practice.
}. This means ZKPs can be used as a way to compress information. One of the
major problems in the blockchain industry is that each (worker) node needs to
store all the information on the chain. With ZKPs we can instead generate a
proof that we have a valid transformation turning the blockchain from a given
state into a new state, committing the proof itself to the blockchain. The full
transactions, on the other hand, can either be kept secret or be posted publicly
in a less consensus-requiring environment, such as IPFS.

Secondly, since zero knowledge provers can both take in public inputs and
private inputs (i.e. the secrets they are proving), we create more granular
control over what is kept private about the transactions we want to submit to
the blockchain. This level of privacy control is necessary in many contexts,
including financial institutions logging their partly private information while
revealing the information that must be reported on.

The final notion that must be discussed in this chapter, is that of Zero
Knowledge Rollups (ZKRs). The idea behind a ZKR is that instead of committing a
proof of a given transaction to L$1$ to update the blockchain's state, we can
roll up a large number of suchs proofs into one proof. This rollup of zero
knowledge proofs can then be committed to the L$1$, significantly reducing the
amount of information each L$1$ (worker) node must store and verify.

\section{What is needed for a ZKR $L2$?}

In this chapter, we will discuss what is needed in order to build a ZKR-based
L$2$ on top of Casper. Such a system consists of six components:
\begin{enumerate}
  \item Contracts
  \item ZK prover
  \item Consensus layer
  \item Data availability layer
  \item Rollup: Compress ZKPs
  \item L$2$ nodes
\end{enumerate}

The first challenge is to turn financial contracts into code. This is solved by
the ACTUS standard, as discussed in section \ref{actus}. This requires us to
provide a Rust implementation of the ACTUS standard, so we can more easily and
securely interact with the standard from a Casper-friendly codebase.

Secondly, we need to implement circuits in a ZK prover to generate and verify
proofs for ACTUS contracts. We are currently exploring different prover systems,
including Halo$2$, Risc$0$, OSL, and Noir. More information on the results of
this exploration will be included in a blogpost written after the exploration
concludes.

The third component is the consensus layer, or Casper's L$1$. Building an L$2$
ZKR requires the L$1$ (worker) nodes to recognize when a transaction comes from
a L$2$ node, and to run the correct ZK verification process on the proof in
order to validate the transaction.

Fourth, a data availability layer takes care of revealing any information about
the transactions which is both allowed to be revealed and not included in the
rollup posted on the L$1$ chain.

Rollup software must be implemented to roll up a large number of ZKPs for
ACTUS-based transactions into one proof.

Finally, the L$2$ itself consists of one or more nodes. These nodes expose an
API in order for submitting ACTUS-based transactions, combine these requests
into an ordered set of transactions, call the ZK prover to generate the proofs
and the ZK rollup to roll them up, and post the resulting proof to the consensus
layer and the set of transactions to the data availability layer. The Casper
Association is considering building an API Gateway in front of the L$2$ nodes in
order to assure good functioning of the second layer and a reasonable
distribution of work.

\section{MVP} \label{mvp}

Now that we discussed the general structure of any L$2$ ZKR, we can dig into
this project's MVP form. The goal of the MVP is to launch a full ACTUS-based
L$2$ ZKR on top of Casper while promoting privacy. We want to allow financial
institutions to log their financial contracts on Casper right from the MVP's
launch. Within these constraints, we want to centralize the service as much as
needed in order to keep the project's scope feasible, taking into account the
Casper Association's resources. The proposed solution to such optimization
problem is as follows:
\bi
  \item Contracts: ACTUS implementation in Rust
  \item ZK prover: Turn ACTUS contract initialization and state updates into
    ZKPs using one ZK prover system
  \item Consensus layer: Intergate with Casper's L$1$
  \item Data availability layer: Centralized append-only dB, read-only
    accessible through a RESTful API
  \item Rollup: Compress ZKPs
  \item L$2$ nodes: The MVP will consist of a single node, both built, deployed
    and maintained by Casper
\ei

The ZK prover and ACTUS code will be available both to the single L$2$ node and
any clients, so they can choose to generate the ZKPs themselves, thereby
preserving their privacy in respect to Casper itself.

\todo{@Mark Is this a reasonable representation of the MVP?}

\section{Long-term vision} \label{vision}

In the long-term, ACTUS on Casper should be sufficiently easy to use and
sufficiently decentralized that financial institutions can replace their core
software with it (e.g. mainframes), and that any risk analyst or regulator can
run analyses and simulations of their choice quickly.

This vision requires two main improvements: The UX for customers and
analysts, and increase decentralization.
\bi
  \item Build an SDK to generate ZKPs for your ACTUS contracts and submit both
    proven and raw contracts to the L$2$.
  \item Build a tool to collect all ACTUS contracts and their current state from
    a given financial institution.
  \item Decentralize the data availability layer, e.g. by working with IPFS.
  \item Decentralize the L$2$. How does consensus work here?
\ei

\todo{@Mark Is this a reasonable representation of the long-term vision? Should
something be mentioned about the API gateway?}

\todo{@Mark Should we mention something about ``if you don't include something
new and interesting in the Casper blockchain, there's no reason for anyone to
adopt Casper''?}

\section{Conclusions}

% Note to self: Trigger the right emotions here.

In this essay, we discussed implementing ACTUS on the Casper blockchain. In the
process, we dug through two major new innovations: The ACTUS standard on
financial contracts and zero knowledge proofs. We linked everything together
into an overview of how to integrate ACTUS with the Casper chain, and explored
why one would be interested in doing so. As it turns out, Casper is in need of a
big innovation to create an explicit selling point for a given audience, growing
its userbase.

\todo{@Mark: Do you want to add a last paragraph here, with some call to action
or such?}

\end{document}
