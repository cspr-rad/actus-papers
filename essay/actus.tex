\documentclass[12pt]{article}

\usepackage{amsfonts}
\usepackage{amsmath}
\usepackage{graphicx}
\usepackage{lmodern}
\usepackage{hyperref}
\usepackage{xcolor}
\definecolor{myblue}{rgb}{0.0, 0.0, 1.0}
\hypersetup{
  colorlinks   = true,
  urlcolor     = myblue,
  linkcolor    = black,
  citecolor    = red
}
\usepackage[document]{ragged2e}
\usepackage{todonotes}

\setlength{\parindent}{0pt}
\usepackage[skip=10pt plus1pt, indent=40pt]{parskip}
\setuptodonotes{inline}

\title{Smart Financial Contracts on Distributed Ledgers}
\author{Nick Van den Broeck, Casper Association}

\begin{document}

\vspace{-1cm}
\maketitle

\tableofcontents
\newpage

\section{Introduction}

In recent years, the blockchain industry has witnessed significant progress in
the realm of smart contracts, particularly within the domain of Decentralized
Finance (DeFi), enabling programmable currencies governed by the principle of
"code is law". However, this paradigm falls short when applied to real-world
financial contracts, which often necessitate legal jurisdiction and enforcement
mechanisms. For instance, consider a loan agreement where the borrower defaults
on payments; in such scenarios, traditional legal recourse becomes essential.
While smart contracts excel in many cases, enforcing obligations tied to legal
concepts like bankruptcy or contractual disputes remains challenging without
overcollateralization.

To address these limitations and unlock a broader array of use cases, the
industry must transition from smart contracts to smart financial contracts.
These are on-chain representations of financial agreements embedded within
specific legal jurisdictions. Essentially, a smart financial contract
encapsulates an agreement from which future cashflows derive, incorporating
jurisdiction-specific mediation.

The development of smart financial contracts hinges on two pivotal innovations:
a standardized representation of financial contracts in code and enhancements in
blockchain transaction throughput and privacy controls. This essay delineates
these innovations and elucidates their synergistic implementation for smart
financial contract realization.

\section{The Importance of Smart Financial Contracts}

Why integrate the ACTUS standard with the Casper blockchain? This question
embodies three distinct facets: the relevance of ACTUS to the financial
industry, the advantages of leveraging blockchain technology, and the
implications for the Casper ecosystem.

Firstly, the financial sector grapples with pervasive issues, characterized by
cycles of regulatory intensity and opacity in risk assessment. Addressing these
challenges necessitates increased transparency and standardization in financial
reporting. The ACTUS protocol offers a solution by standardizing financial
contract representation, enabling comprehensive risk analysis and regulatory
oversight.

Secondly, integrating ACTUS with blockchain technology introduces unparalleled
benefits in privacy, security, and scalability. By leveraging the immutable and
transparent nature of distributed ledgers, financial institutions can enhance
data integrity and streamline compliance processes.

Finally, the integration of ACTUS with Casper presents a strategic opportunity
for the blockchain platform to distinguish itself in the competitive landscape.
By catering to the specific needs of the financial industry and fostering
innovation in smart contract technology, Casper can carve out a unique value
proposition, attracting both developers and enterprises seeking a robust and
compliant blockchain solution.

\section{Use Cases for ACTUS on Casper}

Effective risk analysis hinges on accurate and relevant data. However,
contemporary financial reporting suffers from inherent limitations, including
data opacity and heterogeneity. By adopting the ACTUS standard and integrating
it with Casper, financial institutions can unlock a myriad of use cases, ranging
from decentralized trading platforms to streamlined contract management in
traditional finance.

Central to this endeavor is the concept of homogenized data representation. The
ACTUS standard offers a comprehensive taxonomy of financial contracts, enabling
uniform data reporting and facilitating collaborative risk analysis across
diverse institutions. This standardization fosters transparency and
accountability, mitigating systemic risks and enhancing market resilience.

Moreover, integrating ACTUS with Casper empowers financial institutions to
embrace decentralized finance while preserving privacy and security. By
leveraging blockchain technology, institutions can transact with confidence,
knowing that their financial data remains tamper-proof and auditable.

\section{The ACTUS Standard}

At the core of the integration between ACTUS and Casper lies the ACTUS
standard—a comprehensive framework for representing financial contracts in a
standardized manner. Each financial contract is encoded with a set of terms and
dependencies, culminating in predictable cash flows over time. The ACTUS
taxonomy comprises 32 distinct contract types, each characterized by algorithmic
cash flow generation and comprehensive risk assessment.

Key to the ACTUS standard's efficacy is its ability to fulfill stringent
requirements, including compact expression, compression potential, analytical
robustness, and compatibility with accounting principles. By encapsulating
financial contracts within a standardized framework, ACTUS facilitates
comprehensive risk analysis and regulatory compliance while streamlining
financial reporting processes.

\section{The Need for an L2}

While the Casper blockchain offers a robust foundation for smart contract
execution, the integration of ACTUS necessitates enhancements in transaction
throughput and privacy controls. This imperative underscores the need for Layer
2 (L2) solutions, which augment the capabilities of the underlying blockchain.

Critical to the realization of ACTUS on Casper is the adoption of Zero-Knowledge
Proofs (ZKPs), which offer unparalleled privacy and scalability benefits. ZKPs
enable verifiable computation without disclosing sensitive information, paving
the way for efficient and confidential smart contract execution. By implementing
ZK-based L2 solutions, Casper can accommodate the diverse needs of the financial
industry while maintaining the integrity and security of its blockchain.

\section{Zero-Knowledge Proofs}

Zero-Knowledge Proofs (ZKPs) constitute a cornerstone of privacy-preserving
computation, allowing parties to prove knowledge of a statement without
revealing the underlying information. This cryptographic technique holds immense
promise for blockchain scalability and confidentiality, enabling efficient
transaction processing and secure data transmission.

Central to the efficacy of ZKPs is their ability to compress information and
preserve privacy. By generating succinct proofs of computational integrity, ZKPs
facilitate efficient transaction validation and enable granular control over
data disclosure. Moreover, Zero-Knowledge Rollups (ZKRs) offer a novel approach
to blockchain scalability, aggregating multiple proofs into a single verifiable
statement.

\section{Building an L2 ZKR on Casper}

The implementation of ACTUS on Casper necessitates the development of a robust
Layer 2 Zero-Knowledge Rollup (ZKR) solution. This endeavor encompasses several
key components, including contract representation, ZK proof generation,
consensus integration, data availability, rollup compression, and L2 node
infrastructure.

Central to this initiative is the Rust implementation of the ACTUS standard,
which provides a standardized framework for financial contract representation.
Concurrently, ZK prover systems must be deployed to generate and verify ZK
proofs for ACTUS contracts, ensuring computational integrity and
confidentiality.

Moreover, the consensus layer of Casper must be augmented to accommodate L2
transactions, incorporating ZK proof verification and validation mechanisms.
Similarly, a robust data availability layer is essential for ensuring
transparency and auditability in L2 transaction execution.

Furthermore, the rollup software must be developed to aggregate ZK proofs and
facilitate efficient transaction processing on Casper. Finally, the deployment
of L2 nodes is paramount, providing a scalable and decentralized infrastructure
for ACTUS contract execution and validation.

\section{Minimum Viable Product (MVP)}

The MVP phase of the ACTUS-Casper integration project aims to deliver a fully
functional L2 ZKR solution on the Casper blockchain while prioritizing privacy
and usability. This entails the implementation of ACTUS contracts, ZK proof
generation, consensus integration, and data availability in a centralized
framework.

Key components of the MVP include Rust-based ACTUS implementation, ZK prover
integration, consensus layer augmentation, centralized data availability, rollup
compression, and deployment of a single L2 node. This streamlined approach
ensures rapid deployment and validation of the ACTUS-Casper integration concept,
laying the groundwork for future scalability and decentralization.

\section{Long-Term Vision}

Looking ahead, the long-term vision for ACTUS on Casper revolves around
enhancing usability, scalability, and decentralization. This entails the
development of user-friendly SDKs, decentralized data availability solutions,
and distributed L2 node infrastructure.

Central to this vision is the democratization of financial contract management,
enabling seamless integration with existing systems and fostering innovation in
decentralized finance. By prioritizing user experience and scalability, Casper
aims to position itself as a leading blockchain platform for the financial
industry, offering unparalleled transparency, security, and efficiency.

\section{Conclusion}

In conclusion, the integration of ACTUS with the Casper blockchain represents a
pivotal step towards revolutionizing financial contract management and
compliance. By leveraging standardized contract representation and
Zero-Knowledge Proofs, Casper can address the diverse needs of the financial
industry while maintaining privacy and security.

Moving forward, the Casper ecosystem is poised to drive innovation in smart
financial contracts, offering a robust and scalable platform for decentralized
finance. Through strategic partnerships and continued development, Casper aims
to solidify its position as a leading blockchain solution for the financial
industry, delivering unparalleled transparency, security, and usability.

\end{document}
